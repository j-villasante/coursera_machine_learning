\documentclass{article}
\usepackage{amsmath}

\title{Coursera machine learning}
\date{2020-01-23}
\author{Josue Villasante}

\begin{document}
  \maketitle
  \newpage

  \section{Week 1}
    \subsection{Introduction}
      A computer program is said to learn from experience E with respect to some class of tasks T and performance measure P, if its performance at tasks in T, as measured by P, improves with experience E.
      \\--Tom Mitchell\\\\
      {\large\bfseries Supervised learning}\\
      Those machine learning problems where the relation between the input and the output is known before hand. In other words, the output of the algorithm is given in a known form. This problems can be either regression or classification.\\\\
      {\large\bfseries Unsupervised learning}\\
      Are those machine learning problems where the output of the algorithm is unknown. The most common problem for this type of learning is clustering, where the label by which the algorithm will cluster (or group by) is unknown.
    \subsection{Model and cost function}
      \begin{center}
	      Input variable: $x^{(i)}$\\
	      Output variable: $y^{(i)}$
      \end{center}
\end{document}
